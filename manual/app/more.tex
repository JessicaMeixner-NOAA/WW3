\pagestyle{myheadings} \setcounter{page}{1} \setcounter{footnote}{0}

\section{~Managing multiple model versions} \label{app:more}
\newcounters 
\vssub

\begin{center}
\rule[1mm]{55mm}{1.0mm} WARNING \rule[1mm]{55mm}{1.0mm} \\ \vspace{\baselineskip}
\parbox{120mm}{If version \WWver\ is implemented as an upgrade to previous
versions of \ws, please note that this version may not be compatible with
previous model versions. It is therefore prudent {\it NOT} to install the new
version of \ws\ on top of the old version.} \\
\vspace{\baselineskip} \rule[1mm]{55mm}{1.0mm} WARNING
\rule[1mm]{55mm}{1.0mm}
\end{center}

\noindent
When \ws\ is first installed, the user needs to define a `home' directory for
\ws. This information is stored in {\file .wwatch3.env} in the users home
directory, or locally with the implementation (option selected in installation
script), and is used by virtually all \ws\ utility scripts. If a new model
version is developed or installed, it is prudent to do this in a new
directory, to avoid loss of previous work or issues of possible
incompatibility of model versions. In order to have the proper scripts work
with the proper model version, the user has several basic options.

\begin{itemize}

\item Dynamically update the environment file {\file .wwatch3.env} to point to
      the proper directory in which the present work is done.

\item Use an environment file stored locally with the implementation (option
      introduced in model version \WWver).

\item Point the environment file {\file .wwatch3.env} to a generic directory
      name like {\file wwatch3}, and store various model versions in
      directories with specific names like {\file wwatch3\_3.14} or {\file
      wwatch3\_dev}. Then make the generic name {\file wwatch3} a symbolic
      link to the specific directory to select that directory to work with.

\end{itemize}

\noindent
At NCEP, the second and third method are used, depending on the preferences of
the team member.

\bpage \pagestyle{empty}
