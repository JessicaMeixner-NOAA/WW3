\vsssub
\subsection{~Program design} \label{run:design}
\vssub

The core of \ws\ is the wave model subroutine, which can be
called by either a stand-alone program shell or any other program that
requires dynamically updated wave data. Two such programs are provided with
the \ws\ release (e.g., {\code ww3\_shel} and {\code ww3\_multi}). 
Auxiliary programs include a grid preprocessor {\code ww3\_grid}, a program to
generate artificial initial conditions {\code ww3\_strt}, generic program shells
for individual {\code ww3\_shel} or multi-grid {\code ww3\_multi} applications,
two input pre-processors ({\code ww3\_prep} and {\code ww3\_prnc}), and  
post-processors for gridded ({\code ww3\_outf} and {\code ww3\_ounf}) and point 
({\code ww3\_outp} and {\code ww3\_ounp}) output data.

In this section, note that file names will be identified by the {\file
file} type font, the contents of a file by the {\code code} type font and {\sc
fortran} program elements by the {\F fortran} type font. The main wave model 
subroutine is {\F w3wave}. Data files are identified with the
file extension {\file .ww3}, except in the multi-grid wave model {\file
ww3\_multi}, where the file extension identifies individual grids part of a chosen
multi-grid mosaic. For simplicity, the file extension {\file .ww3} will be used 
throughout this chapter.  

A relational diagram including the basic data flow is presented in 
Fig.~\ref{fig:run_elements}. The figure illustrates a typical workflow, as follows.
 The grid pre-processor {\code ww3\_grid} writes a model definition file {\file mod\_def.ww3} with
bottom and obstruction information and parameter values defining the physical
and numerical approaches. The wave model may have cold or hot starts. 
Hot starting requires a restart file {\file restart.ww3}, created either
by the wave model itself in a previous run, or by the initial conditions program {\code ww3\_strt}.
If a restart file is not available, the wave model will be initialized automatically.
If linear growth or spectral seeding is switch on, the model may start from a flat ocean ($H_s
= 0$), otherwise the initial conditions will consist of a parametric
fetch-limited spectrum based on the initial wind field (see the corresponding
option in the initial conditions program).  

The wave model routine ({\F
  w3wave}) optionally generates up to 9 restart files {\file
  restart{\em{n}}.ww3}, where {\file{\em{n}}} represents a single digit
integer number. For telescoping nest applications, the wave model also 
optionally reads boundary conditions from
the file {\file nest.ww3} and generates boundary conditions for consecutive
nested runs in {\file nest{\em{n}}.ww3}. The model furthermore dumps raw data to the
output files {\file out\_grd.ww3 }, {\file out\_pnt.ww3}, {\file track\_o.ww3}
and {\file partition.ww3} (gridded mean wave parameters, spectra at locations,
spectra along tracks, and partitioned wave data, respectively). The tracks
along which spectra are to be presented is defined in the file {\file
  track\_i.ww3}. Note that the wave model does not write to standard output,
because this would be inconvenient if \ws\ is part of an integrated
model. Instead, it maintains its own log file {\file log.ww3} and optionally a
test output files {\file test.ww3} for a shared memory version of the model,
or {\file test{\em{nnn}}.ww3} for distributed memory versions, where {\em nnn}
is the processor number starting with 1.  Finally, various output
post-processors are available (binary post-processing of raw gridded fields,
point output and track output files; NetCDF and GRIB(2) packing of wave data;
post-processing for later GrADS graphical processing of gridded and spectral
data). A more detailed description of all program elements and their input
files is given below. Note that the source codes of each routine are fully
documented. This documentation is an additional source of information about
\ws.

\setlength{\unitlength}{0.1mm}
\begin{figure}

\begin{picture}(1370,1080)(0,-1080)

\put(  50,  -80){\dashbox{5}(280,80)[c]{{\file grid data}}}
\put( 190,  -80){\vector(0,-1){60}}

\put(   0, -220){\framebox(380,80)[c]{grid preprocessor}}
\put( 190, -220){\vector(0,-1){60}}

\put(  50, -360){\dashbox{5}(280,80)[c]{{\file mod\_def.ww3}}}
\put( 190, -360){\vector(0,-1){380}}
\put( 190, -460){\vector(1,0){60}}
\put( 330, -320){\line(1,0){560}}
\put( 890, -320){\vector(1,1){100}}
\put( 890, -320){\line(1,-1){80}}
\put( 970, -400){\vector(0,-1){160}}

\put( 250, -500){\framebox(280,80)[c]{initial cond.}}
\put( 530, -460){\vector(1,0){60}}

\put( 590, -500){\dashbox{5}(280,80)[c]{{\file restart.ww3}}}
\put( 870, -500){\vector(1,-1){60}}

\put( 590, -640){\makebox(280,80)[c]{{\file restart.ww3}}}
\put( 590, -680){\makebox(280,80)[c]{{\file nest.ww3}}}
\put( 590, -680){\dashbox{5}(280,120)[c]{ }}

\put( 930, -640){\dashbox{10}(280,80)[c]{{\code wave model}}}
\put( 930, -600){\vector(-1,0){60}}
\put( 870, -600){\vector(1,0){60}}

\put( 590, -860){\makebox(280,80)[c]{{\file partition.ww3}}}
\put( 590, -780){\makebox(280,80)[c]{{\file out\_pnt.ww3}}}
\put( 590, -820){\makebox(280,80)[c]{{\file out\_grd.ww3}}}
\put( 590, -860){\dashbox{5}(280,170)[c]{ }}
\put( 970, -780){\vector(-1,0){100}}
\put( 590, -780){\vector(-1,0){60}}

\put( 970, -640){\vector(0,-1){240}}
\put( 830, -960){\makebox(280,80)[c]{{\file log.ww3}}}
\put( 830,-1000){\makebox(280,80)[c]{{\file test.ww3}}}
\put( 830,-1000){\dashbox{5}(280,120)[c]{ }}

\put(1170, -640){\vector(0,-1){100}}
\put(1170, -740){\vector(0,1){100}}
\put(1030, -820){\makebox(280,80)[c]{{\file track\_i.ww3}}}
\put(1030, -860){\makebox(280,80)[c]{{\file track\_o.ww3}}}
\put(1030, -860){\dashbox{5}(280,120)[c]{ }}

\put( 150, -820){\makebox(380,80)[c]{output}}
\put( 150, -860){\makebox(380,80)[c]{postprocessing}}
\put( 150, -860){\framebox(380,120)[c]{ }}

\put(1020, -560){\line(0,1){280}}
\put(1020, -280){\line(1,0){320}}
\put(1210, -620){\line(1,0){130}}
\put(1340, -620){\line(0,1){340}}
\put(1020, -370){\makebox(320,80)[c]{program}}
\put(1020, -410){\makebox(320,80)[c]{shell}}
\put(1020, -450){\makebox(320,80)[c]{or}}
\put(1020, -490){\makebox(320,80)[c]{integrated}}
\put(1020, -540){\makebox(320,80)[c]{program}}

\put(1040,  -80){\dashbox{5}(280,80)[0,0]{{\file input files}}}
\put(1170,  -80){\vector(0,-1){60}}

\put( 990, -220){\framebox(380,80)[c]{input preprocessor}}
\put(1170, -220){\vector(0,-1){60}}

{\scriptsize
\put(  50,-1010){\dashbox{5}(150,40)[0,0]{{\file file}}}
\put( 240,-1010){\dashbox{10}(150,40)[c]{{\code subrout.}}}
\put( 430,-1010){\framebox(150,40)[c]{program}}
\put( 140,-1050){\vector(1,0){60}}
\put( 220,-1070){\makebox(400,40)[l]{data transfer by file}}
\put(   0,-1090){\framebox(630,150){ }}                 }

\end{picture}

\caption{Basic program elements and data flow.}
\label{fig:run_elements}

\botline

\end{figure}


Files specific to \ws\ are opened by name within program subroutines. The unit
numbers, however, are defined by the user\footnote{~Except for {\file
ww3\_multi}.} within each {\file \.inp}, guaranteeing the largest possible 
flexibility for implementation in integrated models.

In addition to the wave model subroutine, an initialization routine and an interface
routine for data assimilation are provided. The routine includes a
generic interface that provides all necessary model components to perform full
spectral data assimilation. This routine is integrated into the generic wave
model shell, which is set up to perform time step managements for a wave model
with or without data assimilation. The shell also provides a simple yet
flexible way to provide the data assimilation scheme with various types of
data. Data assimilation has not yet been included in the multi-grid wave model
shell.
