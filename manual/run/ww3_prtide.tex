\vsssub
\subsubsection{The tide prediction program} \label{sub:ww3prtide}
\vsssub

\proddefH{ww3\_prtide}{w3tide}{ww3\_prtide.ftn}
\proddeff{Input}{ww3\_prtide.inp}{Formatted input file for program.}{10}
\proddefa{mod\_def.ww3}{Model definition file.}{20}
\proddefa{current.ww3\_tide or level.ww3\_tide}{File with tidal constituents.}{user}
\proddeff{Output}{standard out}{Formatted output of program.}{6}
\proddefa{current.ww3 or level.ww3}{Level or current forcing.}{33}

\inpfile{ww3_prtide.tex}

\vspace{\baselineskip} 
\vspace{\baselineskip} 
\noindent 
The user-provided file current.ww3\_tide or level.ww3\_tide is a binary file
that can be obtained by running ww3\_prnc with the 'AT' option and then
renaming the resulting file current.ww3 or level.ww3 into current.ww3\_tide or
level.ww3\_tide . The choice of tidal constituents used for the tidal
prediction can be a subset of the ones present in these files or all of them.

Because of wetting and drying or grid mismatches, the tidal constituents may
be erroneous or absent for some of the \ws\ nodes. The erroneous ones can be
detected using a maximum amplitude on particular components. When the
amplitudes exceeds these maxima, then the tidal constituents are extrapolated
from the nearest nodes. This feature has only been tested on triangular
meshes.

\pb
