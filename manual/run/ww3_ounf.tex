\vsssub
\subsubsection{Gridded output NetCDF post-processor} \label{sec:ww3ounf}
\vsssub

\proddefH{ww3\_ounf}{w3ounf}{ww3\_ounf.ftn}
\proddeff{Input}{ww3\_ounf.nml}{Namelist configuration file.}{10} (App.~\ref{sec:config132})
\proddefa{ww3\_ounf.inp}{Tradition configuration file.}{10} (App.~\ref{sec:config131})
\proddefa{mod\_def.ww3}{Model definition file.}{20}
\proddefa{out\_grd.ww3}{Raw gridded output data.}{20}
\proddefa{NC\_globatt.inp}{Additional global attributes (deprecated)}{994}
\proddefa{ounfmeta.inp}{User defined meta data attributes}{60}
\proddeff{Output}{standard out}{Formatted output of program.}{6}
\proddefa{\opt  .nc}{NetCDF file}{}


\vspace{\baselineskip} 

\noindent
When a single field is put in the file, the abbreviated field name (file
extensions from ww3\_outf) for each data type is given in
Table~\ref{tab:fields} on page~\pageref{tab:fields}.

\noindent
If desired, it is possible to override the default metadata (attributes)
for each variable by providing values in the ounfmeta.inp file.

\noindent
Entries in the ounfmeta.inp file are formatted as follows:
\begin{verbatim}
   META ( [ IFI  IFJ ]  |  FLDID )  [ IFC ]
     attr_name = attr_value
     attr_name = attr_value
     extra_attr = extra_value [type]
   ... repeated as many times as required.
\end{verbatim}

\noindent
An output field is selected using the \texttt{META} keyword followed by
the `field name` or an [IFI, IFJ] integer pair that relates to the
group (IFI) and field (IFJ) flags. See ww3\_shel for details of
field names and flags.

\noindent
Either form may be followed by an integer value (IFC) to select the 
component in multi-component fields (such as wind).

\noindent
Blank lines and comments lines (starting with \$) are ignored.

\noindent
Each \texttt{`attr\_name`} relates to an existing variable attribute and
can be one of the following (altnerative internal name shown in parenthesis):
\begin{itemize}
 \item  standard\_name (varns) [string] : CF Standard name
 \item  long\_name (varnl) [string] : Descriptive long name
 \item  globwave\_name (varng) [string] : Optional GlobaWAVE name
 \item  direction\_reference or dir\_ref (varnd) [string] : Optional reference
 frame for directional fields.
 \item  comment (varnc) [string] : Optional comment.
 \item  units [string] : Units of field
 \item  valid\_min (vmin) [float] : Minimum valid value of data
 \item  valid\_max (vmax) [float] : Maximum valud value of data
 \item  scale\_factor (fsc) [float] : Scaling factor for data - used only when
 variable type is SHORT.
\end{itemize}

\noindent
Additionaly, the following fields are also configurable:
\begin{itemize}
 \item  varnm : the netCDF variable name
 \item  ename : the identifier for single field per file output
\end{itemize}

\noindent
Any other attribute name is assumed to be an optional "extra" 
attribute. This extra attribute can take an optional "type"
keyword to specify the variable tpye of the metadata. If 
no type is supplied, it defaults to a characer type. Valid
types are one of ["c", "r", "i"] for character/string, 
real/float or integer values respectively.

\noindent Some default attibutes can be disabled by setting 
the attribute value to a blank string (""). This will prevent
the attribute from being written in the netCDF file. This
applies only to the \texttt{globawave\_name}, \texttt{comment}
and \texttt{direction\_reference} attributes.

\noindent
Global meta data can be specified with a special \texttt{META global} line.

\noindent
Example ounfmeta.inp file:

\begin{verbatim}
$ Lines starting with dollars are comments.
$ The folowing line starts a meta-data section for the depth field
META DPT
  standard_name = depth
  long_name = "can be quoted string"
  comment = or an unquoted string
  vmax = 999.9

$ Next one is HSig (group 2, field 1)
META 2 1
  $ varns is same as standard_name
  varns = "sig. wave height"
  $ varnl is same as long_name
  varnl = "this is long name"

$ Next one is second component of wind. It also sets an
$ "extra" meta data value (height - a float)
META WND 2
  standard_name = "v-wind"
  height = 10.0 "r"

$ Global metadata:
META global
  institution = UKMO
  comment "space seperated strings should be quoted" c
  version = 1.0 r

\end{verbatim}

\pb
