\vssub
\subsection{~Wave crest and height space-time extremes} \label{sub:space_time_ext}
\conthead{\ws (ISMAR, NCEP)}{Barbariol, F., Benetazzo, A., Alves, J.H.G.M.}

Space-Time (ST) extreme waves are modeled in WAVEWATCH III based on the Euler Characteristics (EC) approach, which states that for a given multi-dimensional (2-D space + time), statistically homogeneous and stationary Gaussian random wave field, the probability of exceedance of the maximal sea surface elevation is approximated by means of the mean value of the EC \citep{Fed12eu}. The ST extreme elevation model used here was formulated by \cite{Fed12sp} for Gaussian sea waves, and extended to second-order nonlinear spatial wave fields by \cite{Fed13} and spatio-temporal fields by \cite{Beet15}. The proposed ST extreme linear model was assessed with numerical simulations \citep{Baet15}, while the extension to second-order nonlinear waves was verified using stereo imaging \citep{Fed13,Beet15}. According to those models, the probability of exceedance of the second-order nonlinear ST maximal crest height $\eta_{2ST_m}$ is approximated (for large threshold $z_2$ with respect to the standard deviation of the surface elevation $\sigma$) as:
\begin{equation}
P(\eta_{2ST_m} > z_2) \approx \left[N_{3D} \left( \frac{z_1}{\sigma} \right)^{2} + N_{2D} \left( \frac{z_1}{\sigma} \right) + N_{1D} \right] \exp{ \left( -\frac{z_1^2}{2 \sigma^2} \right)},
\label{eq_STE1}
\end{equation}
where the nonlinear threshold $z_2$ is related to its linear approximation $z_1$ via the Tayfun quadratic equation using the steepness parameter $\mu$, strictly valid in deep waters, which accounts for bandwidth effects. Parameters $N_{3D}$, $N_{2D}$, and $N_{1D}$ express the average number of 3D, 2D, and 1D waves within the ST region, respectively, and are determined from the moments $m_{ijl}$ of the directional wave spectrum $S(k, \theta)$ defined as follows:
\begin{equation}
m_{ijl}=\int k_{x}^{i}k_{y}^{j}\omega^{l}S(k,\theta)dkd\theta.
\label{eq_STE2}
\end{equation}

The average number of waves in Eq. (\ref{eq_STE1}) also depends on the size of the spatio-temporal domain, namely the spatial dimension $X$ along the mean direction of wave propagation, the spatial dimension $Y$ orthogonal to the mean direction of wave propagation, and the duration $D$. The expected value $\bar{\eta}_{2ST_m}$ (output parameter \textbf{STMAXE}, in meters) of the random variable $\eta_{2ST_m}$ is given by
\begin{multline}
\bar{\eta}_{2ST_m}=\mathbb{E}\left\{\eta_{2ST_m} \right\} = \\ 
	\sigma \left[(h_1+\frac{\mu}{2}h_1^{2})+
\gamma \left( h_1-\frac{2N_{3D}h_1+N_{2D}}{N_{3D} h_1^{2}+N_{2D} h_1+N_{1D}}\right)^{-1} (1+\mu h_1) \right],
\label{eq_STE3}
\end{multline}
where $\gamma \approx 0.5772$ is the Euler-Mascheroni constant, and $h_1$ is the dimensionless (with respect to the standard deviation $\sigma$) most probable (mode) extreme value, which is the largest solution of the implicit equation in $h$
\begin{equation}
\left[N_{3D} h^{2} + N_{2D} h + N_{1D} \right] \exp{ \left( -\frac{h^2}{2} \right)}=1.
\label{eq_STE4}
\end{equation}

The standard deviation $\sigma_{2_m}$ (output parameter \textbf{STMAXD}, in meters) of the crest height $\eta_{2ST_m}$ is given by:
\begin{equation}
\sigma_{2_m}=std(\eta_{2ST_m})=\sigma \frac{\pi}{\sqrt{6}} \left( h_1-\frac{2N_{3D}h_1+N_{2D}}{N_{3D} h_1^{2}+N_{2D} h_1+N_{1D}}\right)^{-1} (1+\mu h_1) .
\label{eq_STE5}
\end{equation}

The expected value of the ST extreme crest-to-trough wave height is obtained using the Quasi-Determinism (QD) model, which predicts the mean shape of ST wave groups close to the apex of their development. According to the QD model the expected value of the crest-to-trough height $\bar{H}_{1_{cm}}$ (output parameter \textbf{HCMAXE}, in meters) of the wave with linear extreme crest height $\bar{\eta}_{1ST_m}$ is expressed as
\begin{equation}
\bar{H}_{1_{cm}}=\mathbb{E}\left\{H_{1_{cm}} \right\} = \bar{\eta}_{1ST_m}(1-\psi_1^* / \sigma^2),
\label{eq_STE6}
\end{equation}
where $\psi_1^* < 0$ is the value of the first minimum of the temporal autocovariance function computed from the spectrum as
\begin{equation}
\psi_1(\tau) = \int S(\omega) \cos{(\omega \tau)} d \omega,
\label{eq_STE7}
\end{equation}
and $\bar{\eta}_{1ST_m}  \psi_1^*/\sigma^2 < 0$ is the expected displacement of the wave trough preceding or following the expected linear extreme crest height $\bar{\eta}_{1ST_m}$, which is computed using Eq. (\ref{eq_STE3}) after letting the wave steepness $\mu=0$. For a given linear group, the height $\bar{H}_{1_{cm}}$ is generally smaller than the maximum expected wave height $\bar{H}_{1_m}$ (output parameter \textbf{HMAXE}, in meters), which is computed as
\begin{equation}
\bar{H}_{1_m}=\mathbb{E}\left\{H_{1_m} \right\} = \bar{\eta}_{1ST_m}\sqrt{2(1-\psi_1^* / \sigma^2)}.
\label{eq_STE8}
\end{equation}

The effect on wave heights of second-order nonlinearities is generally small, particularly in narrow band seas, and it will be neglected in the present implementation to reduce the computational cost. Uncertainty of estimates of $\bar{H}_{1_{cm}}$ (output parameter \textbf{HCMAXD}, in meters) and $\bar{H}_{1_m}$ (output parameter \textbf{HMAXD}, in meters) are determined using the standard deviation of $\eta_{1ST_m}$ ($\sigma_{1_m}$, which is computed using Eq. (\ref{eq_STE5}) after letting the wave steepness $\mu=0$) as follows:
\begin{eqnarray}
std({H}_{1_m})&=&\sigma_{1_m}\sqrt{2(1-\psi_1^* / \sigma^2)}, \nonumber \\
std({H}_{1_{cm}})&=&\sigma_{1_m}(1-\psi_1^* / \sigma^2).
\label{eq_STE9}
\end{eqnarray}

To activate the computation of wave crest and height space-time extremes in WAVEWATCH III, the user has to specify values of the {\F MISC} namelist parameters {\code STDX}, {\code STDY} and {\code STDT} in {\file ww3\_grid.inp} different to -1 (the latter is a default value that avoids computer overheads when these parameters are not wanted). {\code STDX} and {\code STDY} are spatial dimensions over which extremes are calculated. {\code STDT} is the time length over which extremes are calculated. If {\code STDX} and {\code STDY} are left at default values (-1), but {\code STDT} has a namelist value different to default (e.g., greater than 0), then extreme values are provided over time, for a point. Conversely, if {\code STDT} is kept at default (-1) and {\code STDX} and {\code STDY} are greater than zero, instantaneous extreme values are computed over space. When all three parameters are greater than zero, space-time probabilities and values are computed.

Wave crest and height space-time extremes outputs follow the standard WAVEWATCH III parameter framework, and have to be specified as namelists or flags in {\file ww3\_shel.inp} or {\file ww3\_multi.inp}, in which case they are included in the standard gridded binary output files during a model run. Consequently, they also have to be specified in gridded output post-processors for obtaining a final human-readable form. Space-time extremes output parameters available in WAVEWATCH III are provided in Table \ref{tab:ste_parm}.

\begin{table}[h]
\begin{center} \begin{tabular}{|l|c|c|} \hline \hline
Internal Label & User-Interface Label & Description \\ \hline
STMAXE  &   MXE   & Max surface elev (STE) \\
STMAXD  &   MXES  & STD of max crest (STE) \\
HMAXE   &   MXH   & Max wave height (STE) \\
HCMAXE  &   MXHC  & Max wvhgt from crest (STE) \\
HMAXD   &   SDMH  & STD of MXH (STE) \\
HCMAXD  &   SDMHC & STD of MXHC (STE) \\ \hline
\end{tabular}
\end{center}
\caption{User-defined parameters in the computation of wave crest and height space-time extremes.}
\label{tab:ste_parm} \botline \end{table}


