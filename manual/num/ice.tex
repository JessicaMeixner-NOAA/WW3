\vssub
\subsection{~Ice coverage} \label{sub:num_ice}
\conthead{\ws}{H. L. Tolman}

\noindent
Ice covered sea is considered as `land' in \ws, assuming zero wave energy and
boundary conditions at ice edges are identical to boundary conditions at shore
lines. Grid points are taken out of the calculation if the ice concentration
becomes larger than a user-defined concentration. If the ice concentration
drops below its critical value, the corresponding grid point is
`re-activated'. The spectrum is then initialized with a PM spectrum based on
the local wind direction with a peak frequency corresponding to the
second-highest discrete frequency in the grid. A small spectrum is used to
assure that spectra are realistic, even for shallow coastal points.

The above discontinuous ice treatment represents the default model setting in
\ws. In the framework of the modeling of unresolved obstacles as discussed in
\para\ref{sub:num_obst}, a continuous method is also available, as given by
\cite{tol:OMOD03a}. In this method, a user-defined critical ice concentration
at which obstruction begins ($\epsilon_{c,0}$) and is complete
($\epsilon_{c,n}$) are given (defaults are $\epsilon_{c,0} = \epsilon_{c,n} =
0.5$, i.e., discontinuous treatment of ice). From these critical
concentrations, corresponding decay length scales are calculated as,

\begin{equation}
l_0 = \epsilon_{c,0} \min ( \Delta x , \Delta y )
\:\:\: . \label{eq:l0}
\end{equation}
\begin{equation}
l_n = \epsilon_{c,n} \min ( \Delta x , \Delta y )
\:\:\: . \label{eq:ln}
\end{equation}

\noindent
from which cell transparencies in $x$ and $y$ ($\alpha_x$ and $\alpha_y$,
respectively) are calculated as

\begin{equation}
\alpha_x = \left \{ \begin{array}{ccl}
 1 & \mbox{for} & \epsilon \Delta x < l_0 \\
 0 & \mbox{for} & \epsilon \Delta x > l_n \\
\frac{l_n - \epsilon \Delta x}{l_n - l_0} & \multicolumn{2}{c}{\mbox{otherwise}} 
\end{array} \right .
\:\:\: , \:\:\:
\alpha_y = \left \{ \begin{array}{ccl}
 1 & \mbox{for} & \epsilon \Delta y < l_0 \\
 0 & \mbox{for} & \epsilon \Delta y > l_n \\
\frac{l_n - \epsilon \Delta y}{l_n - l_0} & \multicolumn{2}{c}{\mbox{otherwise}} 
\end{array} \right .
\:\:\: . \label{eq:ice_0} 
\end{equation}

\noindent
Details of this model can be found in \cite{tol:OMOD03a}.

Updating of the ice map within the model takes place at the discrete model
time approximately half way in between the valid times of the old and new ice
maps. The map will not be updated, if the time stamps of both ice fields are
identical.

Note that either ice transparency for propagation, or ice as a source term
should be used, but not both approaches. at the same time.
