\vssub
\subsubsection{~Rotated grids} \label{sub:num_space_rotagrid}
\opthead{RTD}{\ws\ (MetOffice)}{J.-G. Li}

\noindent
The rotated grid is a latitude-longitude (lat-lon) grid and is obtained by
rotating the North Pole along a longitude $\lambda_{p}$ to a new position at
latitude $\phi_{p}$ in the standard latitude-longitude system.  The new
pole position is chosen so that the model domain of interest may be placed
around the rotated equatorial area for a evenly-spaced lat-lon mesh. For this
reason the rotated grid is also known as \emph{Equatorial grid}. For instance,
the North Atlantic and European wave (NAEW) model used in the UK Met Office
uses a rotated pole at 37.5N, 177.5E so that London, UK
(\textasciitilde{}51.5N 0.0E) is almost on the rotated equator. This rotated
grid allows a much more evenly spaced lat-lon mesh in the NAEW domain than the
standard lat-lon grid in the same area. 

In \ws, the rotated grid is implemented with minimum changes to the original
lat-lon grid. In fact, the rotated grid is treated just like the standard 
lat-lon grid inside the model. To set up and run a rotated grid model configuration,
users should choose the regular lat-lon grid along with the {\code RTD} switch.
The rotated pole position is set using the {\code PLAT} and {\code PLON} variables
under the namelist {\code ROTD} in the input file {\bf ww3\_grid.inp}
(see Sect.~\ref{sec:config011}). If the pole is set as {\code PLAT = 90.0},
{\code PLON = -180.0}, the grid is treated as a standard lat-lon system.

Model input files, like wind, current and ice files
should be mapped on to the rotated grid. For convenience of nesting in standard
lat-lon grid frameworks, boundary conditions data provided to and output from the
rotated grid use spectra referenced to a standard grid north and standard lat-lon
grid points values, which are converted into rotated grid lat-lon inside \ws.
The list of 2D spectral output locations in {\bf ww3\_shel.inp} or
{\bf ww3\_shel.nml} are also specified in standard lat-lon.

When nesting from a standard grid to a rotated grid model, both the outer and inner
grids should be built with the {\code RTD} switch set when compiling executables for
the models. When nesting from a rotated grid to a standard grid model, the inner
(standard) grid model does not require to be built with {\code RTD}.

Output of spectra at boundary points to one-way nested inner grids are transferred
as described in Appendix~\ref{app:nest}. Output b.c.\ are defined in the input
file (see {\bf ww3\_grid.inp}, Sect.~\ref{sec:config011}) as a sequence of straight
lines given in coordinates of the inner grid. If the inner grid is rotated, it's
pole position must be defined as the values of the array elements
{\code BPLAT(\sl{n})} and {\code BPLON(\sl{n})} under namelist {\code ROTB}
for the outer grid. The array index {\sl{n}} is the index ({\sl{1:9}}) of the boundary
conditions file {\file nest{\sl{n}}.ww3}.

Model directional and x-y vector outputs from a rotated grid can be converted
to a standard grid north reference by setting the UNROT variable in the
{\bf ww3\_grid.inp} namelist ROTD to True. With this set, for point outputs
lat-lon locations all directional
values such as wind direction, current direction and 2D spectra are converted
into standard lat-lon orientation. Functions to de-rotate gridded
fields are applied in {\bf ww3\_ounf}, {\bf ww3\_outf} and {\bf ww3\_grib}. 
When running {\bf ww3\_ounf} and {\bf ww3\_ounp}, the resulting netCDF
files will include a variable attribute direction\_reference, which describes
whether a standard (True North) or rotated grid directional reference frame
has been used. Gridded netCDF files generated by {\bf ww3\_ounf} also include
\emph{standard\_latitude} and \emph{standard\_longitude} two-dimensional arrays
that describe location of the rotated model cell centres in the standard lat-lon
reference frame.

If the user wishes to generate boundary conditions for a rotated pole grid using
the {\bf ww3\_bound} or {\bf ww3\_bounc} boundary processing programs, then it
should be noted that the input spectra for these programs are always expected
to be formulated on a \emph{standard pole}.

Six subroutines are provided in module {\bf w3servmd.ftn} for rotated grid
conversion:
\begin{vlist}
\vit{w3spectn}{}{Turns wave spectrum anti-clockwise by AnglD}
\vit{w3acturn}{}{Turns wave action(k,nth) anti-clockwise by AnglD}
\vit{w3thrtn}{}{Turns direction parameters anti-clockwise by AnglD}
\vit{w3xyrtn}{}{Turns x-y vector parameters anti-clockwise by AnglD}
\vit{w3lltoeq}{}{Convert standard into rotated lat/lon plus AnglD}
\vit{w3eqtoll}{}{Reverse of w3lltoeq, but AnglD unchanged}
\end{vlist}
These subroutines are self-contained and can be extracted outside the model
for pre- or post-processing of rotated grid files.  Some conversion tools have
been developed based on these subroutines but have not been included in \ws\
yet. Refer to the regression test \emph{regtests/ww3\_tp2.11} for an example
of a rotated grid model (NAEW).  Users may find more information in
\emph{smc\_docs/Rotated\_Grid.pdf} or contact Jian-Guo Li for help
(\url{Jian-Guo.Li@metoffice.gov.uk}).
