\vsssub
\subsubsection{~$S_{ice}$: Damping by sea ice (generalization of Liu et al.)} \label{sec:ICE2}
\vsssub

\opthead{IC2}{\ws/NRL}{E. Rogers, S. Zieger, F. Ardhuin}

\noindent
This method for representing the dissipation of wave energy by wave-ice interaction is based on the papers
by \cite{art:LMC88}, \cite{art:LHV91} and \cite{art:Aea15}. The main input ice parameters
is the ice thickness (in meters) that can vary spatially and temporally and is the forcing field  ${C_{ice,1}}$. 

This is a model for attenuation by a
sea ice cover, derived on the assumption that dissipation is caused by
friction in the boundary layer below the ice,
with the ice modeled as a continuous thin elastic plate. The original form by  \cite{art:LMC88} is activated by 
setting the {\code IC2} namelist {\F SIC2} parameter {\code IC2DISPER = .TRUE.}. That form 
 assumes that the boundary layer is always laminar but it uses an eddy viscosity ${\nu}$ that can vary spatially 
and is the forcing field ${C_{ice,2}}$. 

With {\code IC2} and {\code IC3}, the sea ice effects requires solution of a
new dispersion relation. For {\code IC2}, the key equations are:

\begin{equation}\label{eq:ice1}
  {\sigma}^2 = ({gk_r} + {Bk_r^5})/(\coth({k_r}{h_w}) + {k_r}{M}),
\end{equation}
\begin{equation}\label{eq:ice2}
  {c_g} = (g + (5 + 4{k_r}{M}){B}{k_r^5})/(2{\sigma}(1+{k_r}{M})^2),
\end{equation}
\begin{equation}\label{eq:ice3}
  {\alpha} = (\sqrt{{\nu\sigma}}{k_r)}/({c_g}\sqrt{2}(1+{k_r}{M})).
\end{equation}

\noindent
In our notation, $h_w$ is water depth and $h_i$ is ice thickness.  The
variables $B$ and $M$ quantify the effects of the bending of the ice and
inertia of the ice, respectively. Both of these variables depend on $h_i$ 
\citep[see][]{art:LMC88, art:LHV91}.

\vspace{\baselineskip} \noindent
In the case of {\code IC2}, though the ${k_r}$ is calculated, its effect is
not passed back to the main program. The only effect is via ${k_i}$
(dissipation).


 \cite{art:Aea15} distinguish between laminar and 
turbulent regimes, allowing this is activated by setting  {\code IC2DISPER = .FALSE.}. 
In that case the dissipation goes from a laminar form using the molecular viscosity multiplied by an 
empirical adjustment factor {\code IC2VISC} to a turbulent form, amplified by a factor {\code IC2TURB}, for Reynolds numbers 
above a user-defined threshold {\code IC2REYNOLDS}. This transition is smoothed over a range {\code IC2SMOOTH} to take into 
account the random nature of the wave field. In the turbulent regime, the friction factor 
is estimated from a user-specified under-ice roughness length {\code IC2ROUGH}, expected to be of the order of $10^{-4}$~m. 
The parameter {\code IC2TURBS} is an ad hoc enhancement of turbulent dissipation in the Southern hemisphere 
that was introduced for test purposes to investigate sources of bias. This will be deprecated in future versions. It now appears 
that combining IC2 with creep dissipation in IS2 can provide good results for dominant waves in both hemispheres. 
