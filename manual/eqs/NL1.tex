\vsssub
\subsubsection{~$S_{nl}$: Discrete Interaction Approximation (\dia)} \label{sec:NL1}
\vsssub

\opthead{NL1}{\wam\ model}{H. L. Tolman}

\noindent
Nonlinear wave-wave interactions can be modeled using the discrete interaction
approximation \citep[\dia,][]{art:Hea85b}. This parameterization was
originally developed for the spectrum $F(f_r ,\theta)$. To assure the
conservative nature of $S_{nl}$ for this spectrum (which can be considered as
the "final product" of the model), this source term is calculated for
$F(f_r,\theta)$ instead of $N(k,\theta)$, using the conversion
(\ref{eq:jac_fr}).

Resonant nonlinear interactions occur between four wave components
(quadruplets) with wavenumber vector $\bk_1$ through $\bk_4$. In the \dia, it
is assumed that $\bk_1 = \bk_2$. Resonance conditions then require that
%--------------------------%
% Resonance conditions DIA %
%--------------------------%
% eq:resonance
\begin{equation} \left .
\begin{array}{ccc}
  \bk_1 + \bk_2 & = & \bk_3 + \bk_4          \\
  \sigma_2  & = & \sigma_1                   \\
  \sigma_3  & = & (1+\lambda_{nl})\sigma_1   \\
  \sigma_4  & = & (1-\lambda_{nl})\sigma_1
\end{array} \:\:\: \right \rbrace \:\:\: , \label{eq:resonance}
\end{equation}
where $\lambda_{nl}$ is a constant. For these quadruplets, the contribution
$\delta S_{nl}$ to the interaction for each discrete $(f_r,\theta)$
combination of the spectrum corresponding to $\bk_1$ is calculated as
%----------------------------%
% Nonlinear interactions DIA %
%----------------------------%
% eq:snl_dia
\begin{eqnarray}
\left ( \begin{array}{c}
  \delta S_{nl,1} \\ \delta S_{nl,3} \\ \delta S_{nl,4}
\end{array} \right ) & = & D
\left ( \begin{array}{r} -2 \\ 1 \\ 1 \end{array} \right )
C g^{-4} f_{r,1}^{11} \times \nonumber \\
& & \left [ F_1^2
\left ( \frac{F_3}{(1+\lambda_{nl})^4} +
        \frac{F_4}{(1-\lambda_{nl})^4} \right ) -
\frac{2 F_1 F_3 F_4}{(1-\lambda_{nl}^2)^4}
\right ] \: , \label{eq:snl_dia}
\end{eqnarray}
where $F_1 = F(f_{r,1} ,\theta_1 )$ etc. and $\delta S_{nl,1} = \delta
S_{nl}(f_{r,1} ,\theta_1 )$ etc., $C$ is a proportionality constant. The
nonlinear interactions are calculated by considering a limited number of
combinations $(\lambda_{nl},C)$. In practice, only one combination is
used. Default values for different source term packages are presented in
Table~\ref{tab:snl_par}.


% tab:snl_par

\begin{table} \begin{center}
 \begin{tabular}{|l|c|c|} \hline
                    & $\lambda_{nl}$ &     $C$      \\ \hline
ST6                 &      0.25      & $3.00 \; 10^7$  \\ \hline
\wam-3              &      0.25      & $2.78 \; 10^7$  \\ \hline
ST4 (Ardhuin et al.)&      0.25      & $2.50 \; 10^7$  \\ \hline
Tolman and Chalikov &      0.25      & $1.00 \; 10^7$  \\ \hline
\end{tabular} \end{center}
\caption{Default constants in \dia\ for input-dissipation packages.}
\label{tab:snl_par} \botline \end{table}

This source term is developed for deep water, using the appropriate dispersion
relation in the resonance conditions. For shallow water the expression is
scaled by the factor $D$ (still using the deep-water dispersion relation,
however)

%------------------%
% Depth factor DIA %
%------------------%
% eq:snl_shal

\begin{equation}
D = 1 + \frac{c_1}{\bar{k}d} \left [ 1 - c_2 \bar{k} d
\right ] e^{-c_3 \bar{k} d} \: . \label{eq:snl_shal}
\end{equation}

\noindent
Recommended (default) values for the constants are $c_1=5.5$, $c_2=5/6$ and
$c_3=1.25$ \citep{art:Hea85a}. The overbar notation denotes straightforward
averaging over the spectrum. For an arbitrary parameter $z$ the spectral average
is given as

%--------------------------%
% Definition spectral mean %
%--------------------------%
% eq:zbar
% eq:etot

\begin{equation}
\bar{z} = E^{-1} \int_{0}^{2\pi} \int_{0}^{\infty}
z F(f_r,\theta) \; d f_r \; d\theta \: , \label{eq:zbar}
\end{equation}
\begin{equation}
E = \int_{0}^{2\pi} \int_{0}^{\infty}
F(f_r,\theta) \; d f_r \; d\theta \: . \label{eq:etot}
\end{equation}

\noindent
For numerical reasons, however, the mean relative depth is estimated as

%------------%
% kd for DIA %
%------------%
% eq:kd_num
% eq:k_hat

\begin{equation}
\bar{k} d = 0.75 \hat{k} d \: , \label{eq:kd_num}
\end{equation}

\noindent
where $\hat{k}$ is defined as

\begin{equation}
\hat{k} = \left ( \overline{1/\sqrt{}k} \right )^{-2} \: .
\label{eq:k_hat}
\end{equation}

\noindent
The shallow water correction of Eq.~(\ref{eq:snl_shal}) is valid for
intermediate depths only. For this reason the mean relative depth
$\bar{k}d$ is not allowed to become smaller than 0.5 (as in \wam). All
above constants can be reset by the user in the input files of the
model (see \para\ref{sub:ww3grid}).

