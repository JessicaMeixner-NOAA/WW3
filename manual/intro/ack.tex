\vssub
\subsection{~Acknowledgments}
\vssub

The development of \ws\ has been an ongoing process for well over a
decade. The development of WAVEWATCH I was entirely funded through my
Ph.D. work at Delft University. The development of WAVEWATCH II has been
funded entirely through my position as a National Research Council Resident
Research Associate at NASA, Goddard Space Flight Center. The initial
development of WAVEWATCH III version 1.18 was entirely funded by NOAA/NCEP,
with most funding provided by the NOAA High Performance Computing and
Communication (HPCC) office, while I was working as a visiting scientist
through the University Corporation for Atmospheric Research (UCAR).
Developments of version 2.22 have been funded similarly through NOAA/NCEP.
Since then, funding is provided by NCEP, but also by many partners outside
NCEP. Much of the work at NCEP has been performed under contract by Science
Applications International Corporation (SAIC) and IM Systems Group (IMSG) and
their sub-contractors, and by other UCAR visiting scientists.  Special thanks
are due to to The Office of Naval Research (ONR), for the funding of many
model upgrades.

I would finally like to thank all users, collaborators and friends, who have
reported errors and glitches, or have made suggestions for improvements,
particularly those who have alpha- and beta-tested this and previous model
release.

\vspace{\baselineskip}
\vspace{\baselineskip} 
\strut \hfill Hendrik L. Tolman, College Park, February 2014

\vspace{\baselineskip}
\vspace{\baselineskip} 
Support for contributions to  \ws\ outside NOAA/NCEP came under a variety of grants from the Office of Naval Research (in particular 
the 2010-2015 NOPP wave modelling project) the European Union (this includes ERC grant 240009 for IOWAGA,  grant 607476 for SWARP), 
the French 
procurement agency (DGA) funding for projects ECORS and PROTEVS, French ANR LabexMer grant ANR-10-LABX-19-01.